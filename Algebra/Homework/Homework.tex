\documentclass{article}

\usepackage{amsmath,amssymb,graphicx,algpseudocode,algorithm,amsthm}
\usepackage[margin=1in]{geometry}
\usepackage{mathrsfs}
\let\mathcrl\mathscr
\usepackage[mathscr]{euscript}
\usepackage{marginnote}
\usepackage{hyperref}
\usepackage{qtree}
\usepackage{graphicx}
\usepackage{tikz}
\geometry{reversemarginpar}


\author{Benji Altman}

\def\latex{\LaTeX\ }

\def\useLim{\limits}
\newcommand{\question}[1]{\marginnote{#1}}
\let\union\cup
\let\inter\cap
\let\emptyset\varnothing
\let\bigunion\bigcup
\let\biginter\bigcap
\let\composed\circ
\let\cross\times
\def\And{\textit{ and }}
\def\Or{\textit{ or }}
\def\sbSeperator{\,\middle|\,}
\def\Return{\State\textbf{return}\par}
\def\ZNonNegative{{\mathbb Z_{\ge 0}}}
\newcommand{\setcomp}[1]{{#1}^{\mathsf{c}}}
\newcommand{\prodfrom}[3]{\prod\useLim_{#1}^{#2}\LB {#3} \RB}
\newcommand{\sumfrom}[3]{\sum\useLim_{#1}^{#2} \LB {#3} \RB}
\newcommand{\unionfrom}[3]{\bigunion\useLim_{#1}^{#2} \LB {#3} \RB}
\newcommand{\interfrom}[3]{\biginter\useLim_{#1}^{#2} \LB {#3} \RB}
\newcommand{\interacross}[2]{\interfrom{#1}{}{#2}}
\newcommand{\unionacross}[2]{\unionfrom{#1}{}{#2}}
\newcommand{\sumacross}[2]{\sumfrom{#1}{}{#2}}
\newcommand{\prodacross}[2]{\prodfrom{#1}{}{#2}}
\newcommand{\Lim}[3]{\lim\useLim_{{#1} \to {#2}}\LB {#3} \RB}
\newcommand{\set}[1]{\left\{ {#1} \right\}}
\newcommand{\setbuilder}[2]{\left\{{#1} \sbSeperator {#2}\right\}}
\newcommand{\derivative}[2]{\frac{d}{d{#2}}\LB {#1} \RB}
\newcommand{\Exists}[2]{\exists_{#1}\LB {#2} \RB}
\newcommand{\All}[2]{\forall_{#1}\LB {#2} \RB}
\newcommand{\abs}[1]{\left|{#1}\right|}
\newcommand{\card}[1]{\left| {#1} \right|}
\newcommand{\range}[1]{\textit{\textbf{Rng}}\left( {#1} \right)}
\newcommand{\domain}[1]{\textit{\textbf{Dom}}\left( {#1} \right)}
\newcommand{\pset}[1]{\mathcal P\left( {#1} \right)}
\newcommand{\pair}[2]{\left( {#1} , {#2} \right)}
\def\closure{\overline}
\newcommand{\limpts}[1]{{#1} '}
\newcommand{\ooint}[2]{\left( {#1} , {#2} \right)}
\newcommand{\ocint}[2]{\left( {#1} , {#2} \right]}
\newcommand{\coint}[2]{\left[ {#1} , {#2} \right)}
\newcommand{\ccint}[2]{\left[ {#1} , {#2} \right]}
\newcommand{\eqclass}[1]{\bar{#1}}
\newcommand{\ceil}[1]{\left\lceil {#1} \right\rceil}
\newcommand{\floor}[1]{\left\lfloor {#1} \right\rfloor}
\newcommand{\inv}[1]{{#1}^{-1}}
\def\true{\text{True}}
\def\false{\text{False}}
\newcommand{\ball}[2]{B_{#1}\left({#2}\right)}
\def\LB{}
\def\RB{}
\newcommand{\cannonicalSet}[1]{\left[ #1 \right]}
\let\lxor\oplus
\newcommand{\norm}[1]{\left|\left|{#1}\right|\right|}

\newtheorem{theorem}{Theorem}[section]
\theoremstyle{definition}
\newtheorem{definition}{Definition}[section]

\let\setminus-
\let\LB[
\let\RB]
\renewcommand\setcomp[1]{{#1}'}
\renewcommand\question[1]{\marginnote{\textbf{#1}}}


\title{Algebra Homework}


\begin{document}
\maketitle
\tableofcontents

\section{Chapter 1}
\subsection{Section 1}
\subsubsection{Question 1}

Choose $a,b\in S$. We find $$a = a * b = b * a = b$$, and thus all elements in $S$ must be the same element, so there is most one element of $S$.

\subsubsection{Question 2}
Let us choose $a,b,c \in S$.

\question{(a)} We have $$a*b = a - b = -(b-a) = -(b*a)$$, thus iff $0 = a*b = a-b$ we have $a*b = b*a$ as $0=-0$, however for any other value of $a*b$, $a*b\not=b*a$. We also may notice that iff $a=b$, then $a*b=a-b=0$. Thus for all $a\not= b$, $a*b \not=b*a$.

\question{(b)} We have
\begin{align*}
a*(b*c) &= a-(b-c) \\
&= a+(c-b) \\
&= a+c-b \\
&= a-b+c \\
&= a-b-(-c) \\
&= (a-b)-(-c) \\
&= (a*b)*-c
\end{align*} so $a*(b*c) = (a*b)*c$ iff $c=-c$ which is only true if $c = 0$.

\question{(c)} We have $a*0 = a-0 = a$.

\question{(d)} We have $a*a = a-a = 0$.

\subsection{Section 2}

\subsubsection{Question 8}

Let $x \in (A\setminus B) \union (B\setminus A)$ then either $x \in A \setminus B$ or $x \in B\setminus A$. If $x \in A\setminus B$ then we get that $x \in A$ and $x \not\in B$, thus $x \in A\union B$ and $x \not\in A \inter B$, which would mean $x \in (A\union B)\setminus(A\inter B)$. If $x\in B\setminus A$ then we get that $x\in B$ and $x\not\in A$, thus $x\in A \union B$ and $x\not\in A\inter B$, which would mean $x \in (A\union B)\setminus(A\inter B)$. It has now been demonstrated that $(A\setminus B) \union (B\setminus A) \subset (A \union B) \setminus (B\inter A)$.

Now let $x\in(A\union B)\setminus(A\inter B)$. We have that $x \in A\union B$ and $x \not\in A\inter B$. It follows that either $x \in A$ or $x\in B$, however, $x$ is not in both $A$ and $B$. This may be written as: $x \in A$ and $x \not\in B$, or $x\in B$ and $x\not\in A$. This then translates to $x \in A\setminus B$ or $x\in B\setminus A$, therefore, $x\in (A\setminus B) \union (B\setminus A)$. It has now been demonstrated that $(A\union B) \setminus (B\inter A) \subset (A\setminus B)\union (B\setminus A)$.

Now it has been shown that both sets are subsets of each-other, thus $(A\setminus B)\union(B\setminus A) = (A\union B) \setminus (A\inter B)$.



This may be displayed pictorially as follows:

%TODO figure out pictures

\def\secondcircle{(210:0.95cm) circle (1.5cm)}
\def\thirdcircle{(330:0.95cm) circle (1.5cm)}
\begin{tikzpicture}
	\begin{scope}
		\clip \secondcircle;
		\fill[cyan] \thirdcircle;
	\end{scope}
	\draw \secondcircle node [text=black,below left] {$A$};
	\draw \thirdcircle node [text=black,below right] {$B$};
\end{tikzpicture}


\subsubsection{Question 9}

Let $x \in A \inter (B \union C)$, thus $x \in A$ and $x\in B\union C$. We then have that $x \in B$ or $x \in C$. Now as we already know that $x\in A$ then we get that either $x \in B \inter A$ or $x \in C \inter A$ and therefore $x \in (A\inter B) \union (A \inter C)$. Thus it has been shown that $A \inter (B\union C) \subset (A\inter B)\union (A\inter C)$.

Let $x \in (A\inter B) \union (A\inter C)$, thus $x \in (A\inter B)$ or $x\in (A\inter C)$. We then get that either $x \in A$ and $x \in B$ or that $x\in A$ and $x\in C$, either way $x\in A$, thus we may write that $x \in A$ and either $x\in B$ or $x\in C$. This would be the same as $x\in A$ and $x\in B\union C$, which then translates to $x \in A \inter (B\union C)$. Thus it has been shown that $(A\inter B)\union (A\inter C) \subset A\inter (B\union C)$.

We have now shown that both sets are subsets of each-other, thus $A\inter(B\union C) = (A\inter B)\union (A\inter C)$.

\subsubsection{Question 10}

Let $x\in A\union (B\inter C)$, assume then for the sake of contradiction that $x\not\in(A\union B)\inter (A\union C)$. Because $x \in A\union (B\inter C)$ we have that $x \in A$ or $x\in B\inter C$. Because $x\not\in (A\union B)\inter (A\union C)$ we have that $x\not\in A\union B$ or $x\not\in A\union C$. We then get that either $x \not\in A$ and $x\not\in B$ or $x\not\in A$ and $x\not\in C$, either way $x\not\in A$, so we have $x\in B\inter C$. We know that $x\not\in B$ or $x\not\in C$, however we also have that $x\in B$ and $x\in C$ due to $x\in B\inter C$, thus we have a contradiction. Thus $A\union(B\inter C) \subset (A\union B)\inter(A \union C)$.

Let $x\in (A \union B) \inter (A\union C)$ and assume for the sake of contradiction that $x\not\in A\union(B \inter C)$. We then get that $x\not\in A$ and $x\not\in B\inter C$. We also have that $x\in A \union B$ and $x\in A \union C$, so if $x\not\in A$ then we get $x\in B$ and $x\in C$. This is then translated to $x\in B\inter C$ which is a direct contradiction with $x\not\in B\inter C$ and again we have a contradiction. Thus $(A\union B)\inter(A\union C)\subset A\union (B\inter C)$.

We have now shown that both sets are subsets of each other, thus $A\inter(B\union C) = (A\union B)\inter(A\union C)$.

\subsubsection{Question 12}

\question{(a)}
\begin{align*}
	\setcomp{(A\union B)} &= \setbuilder{x\in S}{x\not\in A\union B} \\
	&= \setbuilder{x\in S}{x\not\in A \And x\not\in B} \\
	&= \setbuilder{x\in S}{x \in \setcomp A \And x\in \setcomp B} \\
	&= \setcomp A \inter \setcomp B
\end{align*}

\question{(b)}
\begin{align*}
	\setcomp{(A\inter B)} &= \setbuilder{x\in S}{x\not\in A\inter B} \\
	&= \setbuilder{x\in S}{x\not\in A \Or x\not\in B} \\
	&= \setbuilder{x\in S}{x \in \setcomp A \Or x\in \setcomp B} \\
	&= \setcomp A \union \setcomp B
\end{align*}

\subsubsection{Question 13}
\question{(a)}
\begin{align*}
	A + B &= (A\setminus B) \union (B\setminus A) \\
	&= (B\setminus A) \union (A\setminus B) \\
	&= B + A
\end{align*}

\question{(b)}
First notice that for any set $X$, $X\setminus \emptyset = A$ and that $\emptyset \setminus X = \emptyset$.
\begin{align*}
	A + \emptyset &= (A \setminus \emptyset) \union (\emptyset \setminus A)\\
	&= A \union \emptyset \\
	&= A \\
\end{align*}

\question{(c)}
\begin{align*}
	A\cdot A &= A\inter A\\
	&= A
\end{align*}

\question{(d)}
\begin{align*}
	A + A &= (A\setminus A)\union (A\setminus A) \\
	&= \emptyset \union \emptyset \\
	&= \emptyset
\end{align*}

\question{(e)} To simplify this question let me introduce the logical operation, $a \lxor b$ which is defined as either $a$ or $b$ but not both, and we will show that $a \lxor (b\lxor c) = (a\lxor b)\lxor c$ using truth tables.

\begin{center}
	\begin{tabular}{c|c|c|c|c|c|c}
		$a$&$b$&$c$&$a\lxor b$&$b\lxor c$&$a\lxor(b\lxor c)$&$(a\lxor b)\lxor c$ \\
		\false&\false&\false&\false&\false&\false&\false\\
		\false&\false&\true&\false&\true&\true&\true\\
		\false&\true&\false&\true&\true&\true&\true\\
		\false&\true&\true&\true&\false&\false&\false\\
		\true&\false&\false&\true&\false&\true&\true\\
		\true&\false&\true&\true&\true&\false&\false\\
		\true&\true&\false&\false&\true&\false&\false\\
		\true&\true&\true&\false&\false&\true&\true
	\end{tabular}
\end{center}

Now we wish to show that $A + B = \setbuilder{x\in S}{x \in A \lxor x \in B}$. To do this we will first show that $a\lxor b = (a\land\lnot b) \lor (b\land\lnot a)$, where $\lnot$ is a logical not, $\land$ is a logical and, and $\lor$ is a logical or. We again show this by the following truth table:
\begin{center}
	\begin{tabular}{c|c|c|c|c|c|c|c}
		$a$&$b$&$\lnot b$&$a\land\lnot b$&$\lnot a$&$b\land\lnot a$&$(a\land\lnot b)\lor(b\land\lnot a)$&$a\lxor b$\\
		\false&\false&\true&\false&\true&\false&\false&\false\\
		\false&\true&\false&\false&\true&\true&\true&\true\\
		\true&\false&\true&\true&\false&\false&\true&\true\\
		\true&\true&\false&\false&\false&\false&\false&\false\\
	\end{tabular}
\end{center}
Now we find
\begin{align*}
A + B &= \setbuilder{x\in S}{x\in A+B} \\
&= \setbuilder{x\in S}{x \in (A\setminus B)\union(B\setminus A)}\\
&= \setbuilder{x\in S}{x\in (A\setminus B) \lor x\in(B\setminus A)}\\
&=\setbuilder{x\in S}{(x\in A \land x\not\in B)\lor(x\in B\land x\not\in A)}\\
&=\setbuilder{x\in S}{x\in A\lxor x\in B}\\
\end{align*}
so we then have
\begin{align*}
A+(B+C)&=\setbuilder{x\in S}{x\in A\lxor x\in B+C} \\
&=\setbuilder{x\in S}{x\in A\lxor(x\in B\lxor x\in C)}\\
&=\setbuilder{x\in S}{(x\in A \lxor x\in B)\lxor x\in C}\\
&=\setbuilder{x\in S}{x \in A + B\lxor x \in C}\\
&=(A+B)+C
\end{align*}

\question{(f)}
Suppose $B\not= C$. Because $B\not= C$ there exists some $x\in S$ such that either $x\in B$ and $x\not\in C$ or $x\in C$ and $x\not\in B$, we will assume without loss of generality that $x\in B$ and $x\not\in C$. Now if $x\in A$ then we would find $x\not\in A+B$ and $x\in A+C$. If $x\not\in A$ we would find that $x\in A+B$ and $x\not\in A+C$. We now have shown that $B\not= C\implies A+B\not=A+C$, thus by contrapositive we have $A+B=A+C \implies B=C$.

\question{(g)}
First we will want to show logical equivalence between the statement $a\land(b\lxor c)$ and $(a\land b)\lxor(a\land c)$.
\begin{center}
	\begin{tabular}{c|c|c|c|c|c|c|c}
		$a$&$b$&$c$&$b\lxor c$&$a\land b$&$a\land c$&$a\land(b\lxor c)$&$(a\land b)\lxor(a\land c)$\\
		\false&\false&\false&\false&\false&\false&\false&\false\\
		\false&\false&\true&\true&\false&\false&\false&\false\\
		\false&\true&\false&\true&\false&\false&\false&\false\\
		\false&\true&\true&\false&\false&\false&\false&\false\\
		\true&\false&\false&\false&\false&\false&\false&\false\\
		\true&\false&\true&\true&\false&\true&\true&\true\\
		\true&\true&\false&\true&\true&\false&\true&\true\\
		\true&\true&\true&\false&\true&\true&\false&\false
	\end{tabular}
\end{center}
now we may show
\begin{align*}
A\cdot(B+C) &= A\inter(B+C)\\
&=\setbuilder{x\in S}{x \in A\inter(B+C)}\\
&= \setbuilder{x\in S}{x\in A\land x\in(B+C)} \\
&= \setbuilder{x\in S}{x\in A\land (x\in B \lxor x\in C)}\\
&=\setbuilder{x\in S}{(x\in A\land x\in B)\lxor(x\in A\land x\in C)}\\
&=\setbuilder{x\in S}{x\in A\inter B\lxor x\in A\inter C}\\
&=\setbuilder{x\in S}{x\in (A\inter B)+(A\inter C)}\\
&=(A\inter B)+(A\inter C)\\
&=(A\cdot B)+(A\cdot C)
\end{align*}

\subsubsection{Question 14}
First notice that if $A$ and $B$ are disjoint then $m(A\union B) = m(A) + m(B)$. So now we get the three disjoint sets $A\setminus B$, $A\inter B$, and $B\setminus A$, notice that $A =( A\setminus B) \union (A\inter B)$, that $B =(B\setminus A)\union(A\inter B)$, and $A\union B = (A\setminus B)\union (A\inter B)\union (B\setminus A)$. Now we get $m(A) = m(A\setminus B) + m(A\inter B)$, $m(B) = m(B\setminus A) + m(A\inter B)$, and $m(A\union B) = m(A\setminus B)+(A\inter B)+m(B\setminus A)$. We then get \begin{align*}
m(A) + m(B) &= m(A\setminus B)+m(A \inter B)+m(B\setminus A)+m(A\inter B)\\
&= m(A\union B) + m(A\inter B)\\
m(A) + m(B) - m(A \inter B) &= m(A \union B)
\end{align*}

\subsubsection{Question 22}
\question{(a)} To construct a subset of any set we go through each element and choose to include it or not to, this gives us two possibilities per element. For a set of size $n$ then there are $n$ independent choices to be made in constructing a subset, thus $2^n$ subsets.

\question{(b)} There are exactly $\binom nm = \frac{n!}{m!(n-m)!}$ subsets of a set with $n$ elements that have $m$ elements.

\begin{proof}
	Let us start by defining $\binom nm$ as the number of ways to choose a subset with $m$ elements from a set with $n$ elements. Now we must recognize that $k!$ is the number of ways to order a set with $k$ elements. Then we get that $\binom nmm!(n-m)! = n!$ as we may order our set with $n$ elements by choosing the first $m$ elements in our order ($\binom nm$ possible ways), then ordering those elements ($m!$ ways), and finally ordering the rest of the elements ($(m-n!)$ ways). This gives us $\binom nmm!(n-m)!=n!$ and from there we divide and get $\binom nm=\frac{n!}{m!(n-m)!}$.
\end{proof}

\subsection{Section 3}
\subsubsection{Question 7}
Let $g:S\to T$, $h:S\to T$ and $f:T\to U$ be functions such that $f$ is 1-1 and $f\composed g = f\composed h$. Assume for the sake of contradiction that $g\not=h$, then there exists some $s \in S$ such that $g(s) \not= h(s)$. We know that $f\composed g(s) = f\composed h(s)$, thus $f(g(s)) = f(h(s))$ so $g(s) = h(s)$ by $f$ being 1-1. Thus we have a contradiction and we know that $g=h$.

\subsubsection{Question 8}
\question{(a)} Yes, as all integers are either even or odd and none are both even and odd.

\question{(b)} Let us break this into cases:
\begin{itemize}
	\item If $s_1$ and $s_2$ are even, then there exists $k_1\in\mathbb Z$ and $k_2\in\mathbb Z$ such that $2k_1 = s_1$ and $2k_2 = s_1$. Thus $s_1 + s_2 = 2k_1 + 2k_2 = 2(k_1 + k_2)$, thus $f(s_1 + s_2) = 1$. We also find that $f(s_1) \cdot f(s_2) = 1 \cdot 1 = 1$.
	\item If $s_1$ is even and $s_2$ is odd, then there exists $k_1\in\mathbb Z$ and $k_2\in\mathbb Z$ such that $s_1 = 2k_1$ and $s_2 = 2k_2 + 1$. Thus $s_1 + s_2 = 2k_1 + 2k_2 + 1 = 2(k_1 + k_2) + 1$ so $f(s_1 + s_2) = -1$. We also find that $f(s_1) f(s_2) = 1\cdot -1 = -1$.
	\item If $s_1$ is odd and $s_2$ is even we may write that $f(s_1 + s_2) = f(s_2 + s_1)$ and that $f(s_1)f(s_2) = f(s_2)f(s_1)$ because both addition and multiplication are commutative. Now we see that we have reproduced our previous case and thus in this case the equality holds.
	\item If $s_1$ and $s_2$ are odd, then there exists $k_1\in\mathbb Z$ and $k_2\in\mathbb Z$ such that $2k_1 + 1 = s_1$ and $2k_2 + 1 = s_2$, thus $s_1 + s_2 = 2k_1 + 1 + 2k_2 + 1 = 2(k_1 + k_2 + 1)$ so $f(s_1 + s_2) = 1$. We also find that $f(s_1)f(s_2) = -1\cdot -1 = 1$.
\end{itemize}
Thus for all possible integers $s_1$ and $s_2$, we have $f(s_1 + s_2) = f(s_1)f(s_2)$.

This tells us that even integers are closed under addition. that odd integers added together always are even, and finally that an odd added to an even is odd.

\question{(c)} No, as $f(1\cdot 2) = f(2) = 1$ and $f(1)f(2) = -1 \cdot 1 = -1$.

\subsubsection{Question 12}

\question{(a)} No $f$ is not a function as $2/3 = 4/6$ and $f(2/3) = 2^23^3 \not= 2^43^6 = f(4/6)$.

\question{(b)} We may define $f(m/n) = 2^m3^n$ iff $m$ and $n$ are coprime.

\subsubsection{Question 19}

Let $f(x) = x^2+ax+b$, thus $f'(x) = 2x+a$. $f'(x)$ is linear so there exists only one $x\in\mathbb R$ for which $f'(x) = 0$, and thus this $x$ is a global extrema for $f$, so $f$ can not be surjective. Now consider $x_1 = -\frac a2-1$ and $x_2=-\frac a2+1$, thus
\begin{align*}
f(x_1) &= \left(-\frac a2-1\right)^2+a\left(-\frac a2-1\right)+b \\
&= \frac{a^2}4+2\frac a2+1-\frac{a^2}2-a+b\\
&=\frac{a^2}4+1+b\\
f(x_2) &= \left(-\frac a2+1\right)^2+a\left(-\frac a2+1\right)+b \\
&= \frac{a^2}4-2\frac a2+1-\frac{a^2}2+a+b\\
&= \frac{a^2}4+1+b
\end{align*}

so $f$ must be 1-1.

\subsubsection{Question 23}


\def\RB{}
\def\LB{}

\marginnote{Ugly proof:}
First let us show that there exists some bijection from $\mathbb N$ to $\ZNonNegative^2$. Consider the 1 norm on $\ZNonNegative^2$, defined as $\norm{(a,b)}_1 = a + b$. Then we may partition $\ZNonNegative^2$ into subsets $P_n = \setbuilder{x\in\ZNonNegative^2}{\norm x_1 = n}$, for any $n\in \ZNonNegative$. Notice that for $(a,b)\in P_n$ then $a \le n$ and $b \le n$, thus forcing $P_n$ to be finite. Now we can construct a function mapping from $\mathbb N$ to $\ZNonNegative^2$ by giving each element of $P_1$ a number from $1$ to $\card{P_0}$ (inclusive), then the next $\card{P_1}$ will be given to elements of $P_1$ and so on infinitely. Notice that by construction $x\not= y \implies f(x) \not= f(y)$, so we get this being 1-1, additionally for any $(a,b) \in \ZNonNegative^2$, $(a,b)\in P_{a+b}$ and thus receives a number greater than $\sumfrom{n=0}{a+b-1}{\card{P_n}}$ and less than or equal to $\sumfrom{n=0}{a+b}{\card{P_n}}$. This means that we can label each element of $\ZNonNegative$ with a single natural number and thus have a bijection.

Now we can also construct a trivial bijection, $h:\ZNonNegative^2\to S$ as $h(a,b) = 2^a3^b$. Now we may compose the bijections to get a 1-1 correspondence $\mathbb N = S$ onto $T$.

\bigskip

\marginnote{Nice proof:}
First notice that $T \subset S$ so there exists the trivial injective function from $T$ to $S$. Second notice that $f:S\to T$ defined as $f(s)=2^s$ is both well defined as injective. By the  \href{https://en.wikipedia.org/wiki/Schr\%C3\%B6der\%E2\%80\%93Bernstein_theorem}{Schr\"oder-Bernstein theorem} there must be some bijection from $S$ to $T$.

\subsubsection{Question 28}

Let $S$ be a finite set, with $f:S\to S$. Now let $f(x) = f(y)$, for some $x\not=y$, then there remain $\card S - 2$ elements in $S\setminus\set{x,y}$ and $\card S - 1$ elements in $S\setminus\set{f(x)}$. This means that for any definition of $f$ on $S\setminus\set{x,y}$ it can not possibly be onto $S\setminus\set{f(x)}$. We have now shown $f$ not being 1-1 implies $f$ not being onto, by contrapositive $f$ being onto implies $f$ is 1-1.

\subsubsection{Question 29}

Let $S$ be a finite set, with $f:S\to S$ injective. Now as $f$ is 1-1 each $s\in S$ has a unique $f(s)\in S$, so $f(S)$ must have exactly $\card S$ unique elements, thus $f(S) \subset S$ with exactly $\card S$ elements.\footnote{$f(A)$ is defined as $\setbuilder{y\in\range{f}}{\Exists{x\in \domain{f}}{f(x) = y}}$ when $A \subset \domain{f}$ and $A\not\in\domain{f}$.} Because $S$ is finite, this implies $f(S) = S$.

\subsection{Section 4}

\subsubsection{Question 5}

\question{(a)} First identity:
\begin{align*}
f^2g^2&=ffgg\\
&=f(fg)g\\
&=f(gf)f\\
&=(fg)^2
\end{align*}

\question{(b)} Second Identity: Let $i$ be the identity function.
\begin{align*}
\inv f\inv ggf &= i\\
\inv f\inv ggf\inv{(gf)} &= i\inv{(gf)}\\
=\inv f\inv g i&=i\inv{(fg)}\\
=\inv f\inv g&=\inv{(fg)}
\end{align*}

\subsubsection{Question 9}

\question{(a)} 
\begin{align*}
f^2&:x_1\to x_3, x_2 \to x_4, x_3 \to x_1, x_4 \to x_2\\
f^3&:x_1\to x_4, x_2 \to x_1, x_3 \to x_2, x_4 \to x_1\\
f^4&:x_1\to x_1, x_2 \to x_2, x_3 \to x_3, x_4 \to x_4
\end{align*}

\question{(b)}
\begin{align*}
g^2&:x_1\to x_1, x_2 \to x_2, x_3 \to x_3, x_4 \to x_4\\
g^3&:x_1\to x_2, x_2 \to x_1, x_3 \to x_3, x_4 \to x_4
\end{align*}

\question{(c)}
$$fg:x_1\to x_3, x_2\to x_2, x_3\to x_4, x_4\to x_1$$

\question{(d)}
$$gf:x_1\to x_1, x_2 \to x_3, x_3 \to x_4, x_4 \to x_2$$

\question{(e)}
\begin{align*}
(fg)^3&:x_1\to x_1, x_2 \to x_2, x_3 \to x_3, x_4 \to x_4\\
(gf)^3&:x_1\to x_1, x_2 \to x_2, x_3 \to x_3, x_4 \to x_4
\end{align*}

\question{(f)} No, $fg(x_1) \not= gf(x_1)$ as can be seen above, thus $fg\not=gf$.

\subsubsection{Question 10}
Consider the cycle structure of a permutation $f$. It is obvious that $f^k = i$ if $k$ is the greatest common divisor among all the cycle lengths in $f$. Now for any $f\in S_3$, cycles must be of length one, two, or three. Therefore, as $6 = \gcd(1,2,3)$ for any $f\in S_3$, $f^6=i$.

\subsubsection{Question 14}

Let $F$ be the mapping from $S_m\to S_n$ such that $F(f)$ is defined to be the same as $f$ where $f$ is defined, and acts as the identity elsewhere. Now $F$ is trivially 1-1, so let us show that it satisfies $F(fg) = F(f)F(g)$ for all $f,g \in S_m$. To start let us choose $x$ in the domain of $g$, then $F(g)$ takes $x \to g(x)$ and $F(f)$ takes $g(x) \to fg(x)$, which is obviously the same as what $F(fg)$ does. If $x$ is not in the domain of $g$ then $F(g)$ takes $x\to x$ and $F(f)$ takes $x \to x$ as does $F(fg)$, we can thus conclude that $F(fg) = F(f)F(g)$.

\subsubsection{Question 21}

Let $g_j$ swap $x_1$ and $x_{j+1}$. Now when $n = 1$ this is trivially true as we have $f = i$ which satisfies the definition of $f$. Let us now try and do an induction on this statement. Assume that $g_1g_2g_3\cdots g_{n-1} = f$ when $n$ is some specific fixed constant. Then it follows that for $f'\in S_{n+1}$ where $f'$ is defined just as $f$ was, that is $f': x_1\to x_2, x_2 \to x_3, \ldots, x_n \to x_{n+1}, x_{n+1} \to x_1$, then consider $g_1g_2g_3 \ldots g_n = fg_n$ and this will obviously give us $f'$, so by induction we have shown that this may be done for any $n$.


\subsubsection{Question 27}
For every $b$ in the domain of $f$ there must be exactly one $a$ and $c$ such that $f(a) = b$ and $f(b) = c$. As the domain of $f$ is finite then there must be some $n\in\mathbb N$ such that $f^n(b) = b$. It follows then that if there is some $n$ such that $f^n(s) = t$ then there must also be some $k$ such that $f^k(t) = s$. By symmetry we also know that the converse is true. This means that either $O(s) = O(t)$ or the two are disjoint.

\subsubsection{Question 30}
Each orbit must be exactly of size $1$. This is because otherwise all $n$ such that $f^n = i$, would have to be a multiple of a number that is not $1$, and thus could not be any prime number.

\subsubsection{Question 32}
$g\in A(S)$ commutes with $f$ iff $g$ is closed on the set $\set{x_1,x_2}$.
\begin{proof}
	First we will show by cases that any $g$ that is closed on $\set{x_1,x_2}$ commutes with $f$, then we will show that no other set does so.
	
	\begin{itemize}
		\item Let $s,t \in \set{x_1,x_2}$ with $s\not=t$
		\begin{itemize}
			\item If $g(s) = s$, then $fg(s) = g(t) = t$ and $gf(s) = f(s) = t$.
			\item If $g(s) = t$, then $fg(s) = g(t) = s$ and $gf(s) = f(t) = s$.
		\end{itemize}
		\item Let $s \not\in\set{x_1,x_2}$, then $fg(s) = gf(s)$ as $f$ acts as the identity.
	\end{itemize}
	
	Now if $g$ is not closed on $\set{x_1,x_2}$ then lets say without loss of generality that $g(x_1) = s \not\in\set{x_1,x_2}$ it follows that $fg(x_1) = g(x_2)$ and $gf(x_1) = f(s) = s$. Now $g(x_2) \not= s$ as otherwise both $x_1$ and $x_2$ would map to the same element which is not possible.
\end{proof}

\subsection{Section 5}

\subsubsection{Question 1}
For this we use the Euclidean algorithm, rather then do the somewhat tedious math, I will simply employ a program I have written in Python.
\\
\question{(a)} $(116, -84) = 4 = 8 \cdot 116 + 11 \cdot -84$.
\\
\question{(b)} $(85,65)=5=-3\cdot85+4\cdot65$.
\\
\question{(c)} $(72,26)=2=4\cdot72-11\cdot26$.
\\
\question{(d)} $(72,25)=1,8\cdot72-23\cdot25$.

\subsubsection{Question 4}
This shall be nothing but some simple arithmetic, most of these numbers are factorials making them particularly easy to compute.\\
\question{(a)} $36 = 2^23^2$.\\
\question{(b)} $120 = 2^33^15^1$.\\
\question{(c)} $720 = 2^43^25^1$.\\
\question{(d)} $5040 = 2^43^25^17^1$.

\subsubsection{Question 7}

\question{(a)}First, we write $m = k_1(m,n)$ and $n=k_2(m,n)$ for some $k_1,k_2\in\mathbb Z$. It follows $$\frac{mn}{(m,n)} = k_1k_2(m,n) = mk_2 = nk_1$$ so this satisfies $m|v$ and $n|v$.

\newcommand{\factorize}[2]{\prodacross{{#2}\in\mathbb N}{{p_{#2}}^{{#1}_{#2}}}}

\begin{lemma}
	For $n = \factorize ni$ and $m = \factorize mi$, if $c_i = \min(n_i,m_i)$ then $$(n,m) = \factorize{c}{i}$$ where $p_i$ is the $i^{\text{th}}$ prime number. 
\end{lemma}

\begin{proof}
	For convention we will let $p_i$ be the $i^{\text{th}}$ prime unless otherwise stated. We will also adopt the convention that for any natural number $x$, the sequence $x_i$ will be it's prime factorization, that is $\factorize{x}{i} = x$ unless otherwise stated. Furthermore we will also by convention assume that if a sequence of natural numbers $x_i$ has been defined then $x = \factorize{x}{i}$, unless otherwise stated. As a last note, we will define $\mathbb N = \set{0,1,2,\ldots}$ and $2 = p_0$.
	
	Let $n$ and $m$ be natural numbers, and then let $c_i = \min{n_i,m_i}$ for all $i\in\mathbb N$. We would like to show $c=(n,m)$. First it is trivial that $c>0$.
	
	Second we must show $c|n$ and $c|m$. To do this let $k_i = n_i - c_i$, notice that $n_i \ge c_i$ for all $i$, therefore $k_i$ is an integer for all $i$.
	\begin{align*}
	kc &= \factorize ki \factorize ci \\
	&= \prodacross{i\in\mathbb N}{{p_i}^{n_i-c_i}}\factorize ci \\
	&= \prodacross{i\in\mathbb N}{{p_i}^{n_i}} \\
	&= n
	\end{align*}
	The same argument can be made to show that $c|m$.
	
	Lastly we must show that if $d|n$ and $d|m$ then $d|c$, we will do this by contrapositive, so assume $d\nmid c$, therefore there does not exist any $k$ st. $dk = c$. Further there exists no sequence of natural numbers $k_i$ st. $d\factorize ki = c$. We know have
	\begin{align*}
	\factorize ki \factorize di &= \prodacross{i\in\mathbb N}{{p_i}^{d_i+k_i}} \\
	&\not= \factorize ci
	\end{align*}
	for any sequence $k_i$, therefore there must exists some $i\in\mathbb N$ st. $d_i > c_i$. It follows then that either $d_i > n_i$ or $d_i > m_i$.
\end{proof}

Now note that $\min(a,b) + \max(a,b) = a + b$ for any $a,b$. Therefore if we define $v_i = \max(n_i,m_i)$ and $c_i = \min(n_i,m_i)$ we get
\begin{align*}
\frac{mn}{(m,n)} &= \frac{\factorize mi \factorize ni}{\factorize ci} \\
&= \prodacross{i\in\mathbb N}{{p_i}^{m_i+n_i-c_i}} \\
&= \factorize vi \\
&= v
\end{align*}

Now we just need to show that $v$ is the least common multiple. If $r < v$ and $\factorize ri = r$, it follows that is some $i$ for which $r_i < v_i$, therefore either $m$ or $n$ can not possibly divide $r$ as either $m_i > r_i$ or $n_i > r_i$.

We now know that $mn/(m,n)$ is the least common multiple of $m$ and $n$.

\question{(b)} As we have already shown $v = \prodacross{i\in\mathbb N}{{p_n}^{\max(n_i,m_i)}}$.

\subsubsection{Question 13}

\question{(a)} If $p = 4n$ then $p$ is divisible by four an not prime. If $p = 4n + 2 = 2(2n + 1)$ then $p$ is divisible by two and not odd. Therefore either $p = 4n+1$ or $p = 4n+3$.

\question{(b)} If $p = 6n$ then $p$ is divisible by six and not prime. If $p = 6n + 2 = 2(3n + 1)$ then $p$ is divisible by two and not odd. If $p = 6n + 3 = 3(2n + 1)$ then $p$ is divisible by three and is either the number 3 or is not prime. If $p = 6n + 4 = 2(3n + 2)$ then $p$ is divisible by two. Therefore if $p$ is an odd prime that is not $3$, then either $p = 6n+1$ or $p = 6n+5$.

\subsubsection{Question 17}

Let $p$ be the $n^{\text{th}}$ prime. Assume for the sake of contradiction that there is some $a,b\in\mathbb N$ st. $a^2=pb^2$, and let $\factorize ai = a$ and $\factorize bi = b$. It follows that $\factorize{2a}i=p\factorize{2b}i$. As $p$ is the $n^{\text{th}}$ prime then $$p^{2a_n}\prodacross{i\in\mathbb N\setminus\set n}{{p_i}^{2a_i}} = p^{2b_n+1}\prodacross{i\in\mathbb N\setminus\set n}{{p_i}^{2a_i}}$$ so the prime factorizations can not possibly be the same, so we have a contradiction.

\subsection{Section 6}

\subsubsection{Question 1}

\begin{proof}
Base case, we have $\frac161(1+1)(2\cdot 1 + 1) = \frac166 = 1 = 1^2$, when $n = 1$.
Inductive case we get 
\begin{align*}
\frac 16(n-1)((n-1)+1)(2(n-1)+1) + n^2 &=\frac16(n-1)n(2n-1) + n^2 \\
&= \frac16\left(2n^3 - 3n^2+n\right) + n^2 \\
&= \frac16\left(2n^3 + 3n^2+n\right) \\
&= \frac16n(2n^2+3n + 1) \\
&= \frac16n(n+1)(2n+1)
\end{align*}
\end{proof}

\subsubsection{Question 2}

\begin{proof}
	Base case, we have $\frac141^2(1+1)^2 = \frac144 = 1 = 1^3$, when $n = 1$. Inductive case we get 
	\begin{align*}
	\frac14(n-1)^2((n-1)+1)^2 + n^3 &= \frac14n^2(n-1)^2 + n^3 \\
	&= \frac14\left(n^4-2n^3+n^2\right)+n^3\\
	&= \frac14\left(n^4+2n^3+n^2\right)\\
	&= \frac14n^2(n+1)^2
	\end{align*}
\end{proof}

\subsubsection{Question 8}

\begin{proof}
Our base case is trivial when $n=1$. In our inductive case we get
\begin{align*}
\frac{(n-1)}{n} + \frac1{n(n+1)} &= \frac{(n-1)(n+1) + 1}{n(n+1)}\\
&= \frac{n^2}{n(n+1)} \\
&= \frac n{n+1}
\end{align*}
\end{proof}


\subsubsection{Question 14}

Let $n = 0$, then it is trivial that $n^p - n$ is divisible by $p$ for any prime $p$.

Now let $n$ be a fixed non-negative integer, and assume that $n^p-n$ is divisible by $p$ for any prime $p$. By the binomial theorem we have
\begin{align*}
(n+1)^p - (n+1)
&= \sumfrom{i=0}{p}{\binom{p}{i}n^i}-n-1 \\
&= \sumfrom{i=1}{p-1}{\binom{p}{i}n^i}+n^p+1-n-1 \\
&= \sumfrom{i=1}{p-1}{\binom{p}{i}n^i}+\left(n^p-n\right)
\end{align*}
By our assumption we have $n^p-n$ is divisible by $p$. Additionally $\binom pi$ must be divisible by $p$ for all $0 < i < p$ because $\binom pi = \frac{p!}{i!(p-i)!}$ and $p$ is prime.

By induction we then know that $n^p-n$ is divisible by $p$ for any prime $p$.

\subsection{Section 7}
\subsubsection{Question 1}

\question{(a)} $(6-7i)(8+i) = 48 - 56i + 6i + 7 = 55-50i$

\question{(b)} $(\frac 23 + \frac 32i)(\frac23 - \frac 32i) = \frac49+\frac94=\frac{16+81}{36} = \frac{97}{36}$

\question{(c)} $(6-7i)(8-i) = 48-56i-6i-7 = 41-62i$

\subsubsection{Question 2}

In general $z^{-1} = \frac{\bar z}{\abs{z}^2}$

\question{(a)} $z^{-1} = \frac{6}{6^2+8^2} - \frac8{6^2+8^2}i$

\question{(b)} $z^{-1} = \frac{6}{6^2+8^2} + \frac8{6^2+8^2}i$

\question{(c)} $z^{-1} = \frac1{\sqrt2} - \frac1{\sqrt 2}i$

\subsubsection{Question 3}
Using Lemma 1.7.1, the fact that $\overline 1 = 1$, and some group axioms, we get.
\begin{align*}
1 &= (\overline z)^{-1}\overline z \\
\therefore \overline 1 &= \overline{(\overline z)^{-1}\overline z} \\
&= \overline{(\overline z)^{-1}}\cdot\overline{(\overline z)} \\
&= \overline{(\overline z)^{-1}} \cdot z \\
\therefore z^{-1} &= \overline{(\overline z)^{-1}} \\
\therefore \overline{z^{-1}} &= \overline{\left(\overline{(\overline z)^{-1}}\right)} \\
&= (\overline z)^{-1}
\end{align*}

\subsubsection{Question 6}

For any $z \in\mathbb C$, there exists $a,b\in\mathbb R$ such that $z = a+bi$. Now $\overline z = a-bi$ by definition. Therefore $z = \overline z$ iff $b=0$ as $a-bi = a+bi$ iff $b=0$. Finally if $b=0$ then $z = a$ and therefore $z$ is real, if $b\not=0$ then $z=a+bi$ for some non-zero $b\in\mathbb R$ so $z$ has an imaginary part and is not real. Therefore we have shown that $z = \overline z$ iff $z\in\mathbb R$.

Now if $a=0$ the we say that $z$ is purely imaginary as there is no real part to $z$. So if $z$ is purely imaginary then $z = bi$ and $\overline z = -bi = -z$. If we start with $\overline z = -z$ then we get 
\begin{align*}
-(a+bi) &= a-bi \\
-a-bi &= a-bi \\
-a = a\\
a = 0
\end{align*}
so $z$ must be purely imaginary. Putting this all together we get that $-z=\overline z$ iff $z$ is purely imaginary.



%\subsubsection{Question 8}

%By lemma 1.7.3 from the book we know that $\abs{uv} = \abs u\abs v$, therefore, as $zz^{-1} = 1$ we also have $\abs{zz^{-1}} = 1$ so $\abs z\abs{z^{-1}} = 1$ and it follows $\abs {z^{-1}} = 1 / \abs z$.

\subsubsection{Question 11}

\question{(a)} $z = \cos\frac{7\pi}4 + i\sin\frac{7\pi}{4}$

\question{(b)} $z = 4\left(\cos \frac\pi2 + i\sin \frac\pi2\right)$

\question{(c)} $z = 36\left(\cos \frac\pi4 + i\sin \frac\pi4\right)$

\question{(d)} $z = 13\left(\cos \frac{2\pi}3 + i \sin\frac{2\pi}3 \right)$

\subsubsection{Question 13}
\begin{align*}
\left(\frac12 + \frac12\sqrt3i\right)^3 &= \left(\frac12\left(1+\sqrt3i\right)\right)^3 \\
&= \frac18\left(1+\sqrt3i\right)^3 \\
&= \frac18\left( 1 + 3\sqrt3i+ 3\left(\sqrt3i\right)^2 + \left(\sqrt3i\right)^3 \right) \\
&= \frac18\left( 1 + 3\sqrt3i - 9 - 3\sqrt3i \right) \\
&= \frac18\left(-8\right) \\
&= -1
\end{align*}


%\subsubsection{Question 14}

%Get help from Russos?
%By Euler's equation we know that $\cos\theta + i\sin\theta = e^{i\theta}$. We have ${\left(e^{i\theta}\right)}^m = e^{im\theta} = \cos (m\theta) + i\sin (m\theta)$

\subsubsection{Question 20}

Let us adopt the notation that for any $c\in\mathbb C$, $$c = c_a + c_bi$$ where $c_a,c_b\in\mathbb R$.
\begin{align*}
\abs{z+w}^2+\abs{z-w}^2 &= \abs{z_a+z_bi + w_a+w_bi}^2 + \abs{z_a+z_bi - w_a-w_bi}^2 \\
&= (z_a+w_a)^2+(z_b+w_b)^2+(z_a-w_a)^2+(z_b-w_b)^2 \\
&= {z_a}^2+2z_aw_a+{w_a}^2 + {z_b}^2+2z_bw_b+{w_b}^2 +{z_a}^2-2z_aw_a+{w_a}^2 + {z_b}^2-2z_bw_b+{w_b}^2 \\
&= 2\left({z_a}^2+{a_b}^2+{w_a}^2+{w_b}^2\right)\\
&= 2\left( \abs{z}^2 + \abs{w}^2 \right)
\end{align*}

\subsubsection{Question 21}

Our approach here is to partition $A$ into countably many finite sets, this will show that there is a 1-1 and onto correspondence from $A$ to $\mathbb N$ as they are both countably infinite. We define $\abs{a+bi}_1 = \abs a + \abs b$. Now we define the set $Z_k = \setbuilder{z\in A}{\abs{z} = k}$ for $k \in\mathbb N \union\set0$. Now for all $z\in A$, there exists some $k\in\mathbb N \union\set 0$ such that $z\in Z_k$ as for any $z\in A$, $z = a+bi$ for $a,b\in\mathbb Z$ and therefore $\abs{z} = \abs{a} + \abs{b} \in \mathbb N \union \set 0$ so there is some $k\in\mathbb N \union\set 0$ such that $z \in Z_k$. For any $k\in\mathbb N\union\set 0$ we also have $Z_k$ is finite as for all $a+bi \in Z_k$, $\abs a\le k$ and $\abs b \le k$, therefore there are only finitely many possibilities for $a$ and $b$. Now we have $\unionacross{k\in\mathbb N \union\set 0}{Z_k} = A$ with each $Z_k$ finite, so $A$ must be countable.


\subsubsection{Question 22}	

First we will prove that $P(\overline x) = \overline{P(x)}$ for any polynomial $P:\mathbb C \to \mathbb C$ with real coefficients, $\alpha_0, \alpha_1,\ldots,\alpha_n$. Let $z \in \mathbb C$ such that $z = a+bi$ with $a$ and $b$ real. Notice that for any $\alpha\in\mathbb R$, 
\begin{align*}
\alpha\overline z &= \alpha(a - bi) \\
&= \alpha a - \alpha bi \\
&= \overline{\alpha z}
\end{align*}
From lemma 1.7.1 we get $\overline{zw} = \overline z \overline w$, so it follows that $\overline{z^n} = \overline{z}^n$. Finally from lemma 1.7.1 we also get $\overline {z + w} = \overline z + \overline w$ so it follows that $\overline{\sum z_j} = \sum \overline{z_j}$. So if $P(x) = \sumfrom{j=0}{n}{\alpha_jx^j}$ then it follows
\begin{align*}
P(\overline{x}) &= \sumfrom{j=0}{n}{\alpha_j\overline x^j} \\
&= \sumfrom{j=0}{n}{\alpha_j\overline{x^j}} \\
&= \sumfrom{j=0}{n}{\overline{\alpha_jx^j}} \\
&= \overline{\sumfrom{j=0}{n}{\alpha_jx^j}} \\
&= \overline{P(x)}
\end{align*}

Therefore if we have any polynomial $P$ with real coefficients, and $P(x) = 0$ then $P(\overline x) = \overline 0 = 0$.


\section{Chapter 2}
\subsection{Section 1}
\subsubsection{Question 8}
Let us start with when $n=0$, then $(a*b)^n = e = a^n*b^n$.

For $n > 0$ we will do induction, so let us assume that $(a*b)^{n-1} = a^{n-1}*b^{n-1}$, therefore 
\begin{align*}
(a*b)^n &= (a*b)^{n-1}*(a*b) \\
&= (a^{n-1}*b^{n-1}) * (a * b) \\
&= (a^{n-1} * a) * (b^{n-1} * b) \\
&= a^n * b^n
\end{align*}
as we already have the case $n=0$ this induction proves the statement for $n\ge 0$.

Now assume $n<0$, therefore $a^n = \left(a^{-1}\right)^{-n}$ and as $a^{-1} \in G$ and $-n > 0$ then we simply refer to our previous work and conclude that the statement still holds.

\subsubsection{Question 9}

Let $a,b\in G$.
\begin{align*}
e &= (a*b)^2 \\
e &=  a^2 \\
e &= b^2 \\
e &= e * e \\
&= a^2 * b^2\\
a^2 * b^2 &= (a*b)^2\\
a*a*b*b &= a*b*a*b \\
a^{-1}*a*a*b*b*b^{-1} &= a^{-1}*a*b*a*b*b^{-1}\\
a*b &= b*a
\end{align*}

\subsubsection{Question 19}

We simply list off all elements of $S_3$ as $S_3$ is small.

\begin{tabular}{c|c|c}
	$x\in S_3$ & Does $x^2 = e$ & Does $x^3 = e$ \\
	\hline
	$(1,2,3)$ & Yes & Yes \\
	$(1,3,2)$ & Yes & No \\
	$(2,1,3)$ & Yes & No \\
	$(2,3,1)$ & No & Yes \\
	$(3,2,1)$ & Yes & No \\
	$(3,1,2)$ & No & Yes \\
\end{tabular}

\subsubsection{Question 20}
This is all elements $p\in S_4$ such that there does not exist exactly one $x$ such that $p(x) = x$.
\begin{enumerate}
	\item $(1,2,3,4)$
	\item $(1,2,4,3)$
	\item $(2,1,3,4)$
	\item $(2,1,4,3)$
	\item $(1,4,3,2)$
	\item $(3,2,1,4)$
	\item $(3,4,1,2)$
	\item $(1,3,2,4)$
	\item $(4,2,3,1)$
	\item $(4,3,2,1)$
	\item $(2,3,4,1)$
	\item $(2,4,1,3)$
	\item $(3,4,2,1)$
	\item $(3,1,4,2)$
	\item $(4,1,2,3)$
	\item $(4,3,1,2)$
\end{enumerate}
\subsubsection{Question 26}

Let $G$ be a finite group. Assume for the sake of contradiction that there is some $a\in G$ such that for all $n \in\mathbb N$, $a^n\not= e$. As $G$ is finite there then must be some $n_1\not=n_2$ such that $a^{n_1} = a^{n_2}$ by the pigeon hole principle. Let us assume without loss of generality that $n_2 > n_1$, it follows then that $a^{n_1}*a^{-n_1} = a^{n_2}*a^{-n_1}$ and therefore $e = a^{n_2 - n_1}$. This means we have a contradiction and therefore there exists some $n\in\mathbb N$ such that $a^n=e$ for any $a \in G$.

\subsubsection{Question 27}

We already have shown that each element $a\in G$ has some specific $n_a$ such that $a^n = e$. It follows then that if $m = {\prodacross{a\in G}{n_a}}$ then $a^m = e$ for all $a\in G$.
\begin{proof}
	Choose $a \in G$ and let $m = \prodacross{g\in G}{n_g}$. Now $a^m = a^{(m/n_a)(n_a)}$, and for notation let $m_a = \frac m{n_a}$. We now have
	\begin{align*}
	a^m &= a^{n_a\cdot m_a} \\
	&= \left(a^{n_a}\right)^{m_a} \\
	&= e^{m_a} \\
	&= e
	\end{align*}
\end{proof}

\subsubsection{Question 28}

First we know that for any $a \in G$ there exists some $a^{-1} \in G$ such that $a^{-1} a = e$ We will adopt this notation as well as simply saying $ab = a * b$ for $a,b\in G$. Now we have
\begin{align*}
aa^{-1}aa^{-1} &= aea^{-1} \\
&= aa^{-1} \\
\therefore \left(aa^{-1}\right)^{-1}aa^{-1} &= \left(aa^{-1}\right)^{-1}aa^{-1}aa^{-1} \\
\therefore e &= aa^{-1}
\end{align*}

Next we wish to prove some a lemma. If $ab = ac$ then $b = c$
\begin{proof}
	\begin{align*}
	ab = ac &\implies a^{-1}ab = a^{-1}ac \\
	&\implies eb=ec \\
	&\implies b = c
	\end{align*}
\end{proof}

Now with this we can say that for all $a \in G$, there exists exactly one inverse as if $ab = e = ac$, then $b = c$. We also can say an element $a\in G$ is the inverse of exactly one element by the exact same proof.

Finally we get
\begin{align*}
aea^{-1} &= aa^{-1} \\
&= e \\
\therefore a^{-1} &= (ea)^{-1} \\
\therefore a &= ea
\end{align*}
	

\subsection{Section 2}
\subsubsection{Question 1}
\subsubsection{Question 2}
\subsubsection{Question 5}

\subsection{Section 3}
\subsubsection{Question 4}
\subsubsection{Question 5}
\subsubsection{Question 11}
\subsubsection{Question 22}
\subsubsection{Question 24}
\subsubsection{Question 26}
\subsubsection{Question 28}
\subsubsection{Question 29}

\subsection{Section 4}
\subsubsection{Question 1}
\subsubsection{Question 5}
\subsubsection{Question 18}
\subsubsection{Question 24}
\subsubsection{Question 30}
\subsubsection{Question 35}
\subsubsection{Question 37}
\subsubsection{Question 38}
\subsubsection{Question 42}
\subsubsection{Question 43}

\subsection{Section 5}
\subsubsection{Question 3}
\subsubsection{Question 17}
\subsubsection{Question 18}
\subsubsection{Question 19}
\subsubsection{Question 24}
\subsubsection{Question 26}
\subsubsection{Question 29}
\subsubsection{Question 38}
\subsubsection{Question 42}

\end{document}

