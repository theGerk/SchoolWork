\documentclass{article}

\title{Maximal Frequent Set Algorithm}
\author{Benji Altman}

\usepackage{amsmath,amssymb,graphicx,algpseudocode,algorithm} %load extra symbols and environments
\usepackage[margin=1in]{geometry} %set margins
\usepackage{mathrsfs}

\let\union\cup
\let\inter\cap
\let\cross\times
\let\emptyset\varnothing
\let\bigunion\bigcup
\let\true\top
\let\false\bot
\let\biginter\bigcap

\let\setminus-
\let\bigcross\bigtimes
\def\Return{\State\textbf{return} }
\newcommand{\setcompliment}[1]{{#1}^{\mathsf{c}}}
\newcommand{\unionfrom}[3]{\bigunion\limits_{#1}^{#2} \left( {#3} \right)}
\newcommand{\interfrom}[3]{\biginter\limits_{#1}^{#2} \left( {#3} \right)}
\newcommand{\crossfrom}[3]{\bigcross\limits_{#1}^{#2} \left( #3 \right)}
\newcommand{\productfrom}[3]{\prod\limits_{#1}^{#2} \left( #3 \right)}
\newcommand{\sumfrom}[3]{\sum\limits_{#1}^{#2} \left( {#3} \right)}
\newcommand{\set}[1]{\left\{ {#1} \right\}}
\newcommand{\setbuilder}[2]{\set{#1 | #2}}
\newcommand{\derivative}[2]{\frac{d}{d{#2}}\left( {#1} \right)}
\newcommand{\Exists}[2]{\exists_{#1}\left( {#2} \right)}
\newcommand{\abs}[1]{\left|{#1}\right|}
\newcommand{\cardinality}[1]{\overline{\overline{#1}}}
\newcommand{\interacross}[2]{\interfrom{#1}{}{#2}}
\newcommand{\unionacross}[2]{\unionfrom{#1}{}{#2}}
\let\times\cdot


\newcommand{\CommentLine}[0]{\State\textbackslash\textbackslash}
\begin{document}
\maketitle
\section{Pseudocode}
Let $P(S)$ be read as the statement ``$S$ is a special set'', and let $\mathcal U$ be any finite superset of $\left\{x: P\left(\{x\}\right)\right\}$.

\begin{algorithm}
	\caption{depth-first searth}\label{dfs}
	\begin{algorithmic}[1]
		\Function{FindMaximalSpecialSet}{\null}
			\Return \Call{GetSpecialSuperset}{$\mathcal U,\emptyset,\emptyset$}
		\EndFunction
		\\
		\Function{GetSpecialSuperset}{$A,B,\mathcal O$}
			\CommentLine Step 1: thin out array
			\ForAll{$x \in A$}
				\If{$\neg$ \Call {$P$}{$B \union \set x$}}
					\State $A \gets A \setminus \set x$
				\EndIf
			\EndFor
			\\
			\CommentLine Step 2: add to output, or call recursive
			\If{$A=\emptyset$}
				\\
				\CommentLine If $B$ is not a subset of any set in $\mathcal O$ add it to $\mathcal O$ and return.
				\ForAll{$O \in \mathcal O$}
					\If{$B \subseteq O$}
						\Return $\mathcal O$ \Comment{Ends the function if the return statement is hit.}
					\EndIf
				\EndFor
				\Return $\mathcal O \union \set B$
			\Else
				\\
				\CommentLine Make recursive calls, moving elements from $A$ into $B$.
				\ForAll{$a \in A$}
					\State $A \gets A \setminus \set a$
					\State $\mathcal O \gets $ \Call{GetSpecialSuperset}{$A,B\union\set a,\mathcal O$} \Comment{$\mathcal O \subseteq$ \Call{GetSpecialSuperset}{$A,B,\mathcal O$}}
				\EndFor
				\Return $\mathcal O$
			\EndIf
		\EndFunction
	\end{algorithmic}
\end{algorithm}

\section{Explanation}


\subsection{details}
We are going to tackle this problem with recursion, so we are going to try and show that our function ``GetSpecialSuperset'' will return the set of maximal special supersets of $B$, if $B$ is a special set and $A$ is a finite superset of the set of all elements which are special with $B$.\footnote{$x$ is special with $S$ iff $S \union \set x$ is special and $x\not\in S$.}

We start by limiting $A$ to only being the set of elemetns with are special with $B$. If there are none, then $A$ will become the empty set, and we would know that $B$ is the only maximal special superset of $B$, and thus return the set $\set B$.

If $A$ is not the empty set (there are elements that are special with $B$), then for each element $x\in A$ we get the set, $\mathcal R$, of maximal special supersets of $B \union \set x$. $\mathcal R$ is found by applying this function again with the new $A$ set as our current $A$, just without $x$, and the new $B$ set as $B \union \set x$. We can see this is true as if some $k$ is speical with $B \union \set x$ then it must also be special with $B$, $A$ is the set of all things special with $B$ and $A \setminus \set x$ is the set of all things special with $B$ except $x$, which we know is in $B \union \set x$, thus $A\supseteq \setbuilder{k}{k \text{ is special with } B \union \set x}$, which is our requirement for what we need $A$ to be. We also know that our new $B$ is special by defintion of $x$ being special with $B$. To take things one step further, we can remove $x$ from our current $A$ as if we have all maximal supersets $B \union \set x$ then we never need to check $B$ with $x$ again as it would have already been found.\footnote{This justifies line 18.} Once we have found $\mathcal R$ we add all sets in $\mathcal R$ to $\text{Output}$.

Once we have iterated through all $x\in A$ we return $\text{Output}$ as it now contains all supersets of $B$ which are special.

\subsection{example}

Let $\mathcal U=\set{0,1,2,3}$ and our set of maximal special sets, the expected output we will call $\mathcal T$ and we will let $\mathcal T = \set{\set{0,2},\set{1}}$, thus $P(S)=\Exists{E\in\mathcal T}{S\subseteq E}$.

\def\ind{\phantom{$\rightarrow$}\hspace{1em}}
\def\nnd{$\rightarrow$\hspace{1em}}
\vspace{4ex}
\noindent
$A: \set{0,1,2,3}$\\
$B: \set{}$\\
$\mathcal O: \set{}$\\
$x:0$\\
$P(B\union\set x)$\\
$x:1$\\
$P(B\union\set x)$\\
$x:2$\\
$P(B\union\set x)$\\
$x:3$\\
$\neg P(B\union\set x)$\\
$A:\set{0,1,2}$\\
$A\not=\emptyset$\\
$a:0$\\
$A:\set{1,2}$\\
\nnd$A:\set{1,2}$\\
\ind$B:\set0$\\
\ind$\mathcal O:\set{}$\\
\ind$x:1$\\
\ind$\neg P(B\union\set x)$\\
\ind$A:\set{2}$\\
\ind$x:2$\\
\ind$P\set{0,2}$\\
\ind$A\not=\emptyset$\\
\ind$a:2$\\
\ind$A:\set{}$\\
\ind\nnd$A:\set{}$\\
\ind\ind$B:\set{0,2}$\\
\ind\ind$\mathcal O:\set{}$\\
\ind\ind$A=\emptyset$\\
\ind$\mathcal O:\set{\set{0,2}}$\\
$\mathcal O:\set{\set{0,2}}$\\
$a:1$\\
$A:\set2$\\
\nnd$A:\set{2}$\\
\ind$B:\set{1}$\\
\ind$\mathcal O:\set{\set{0,2}}$\\
\ind$x:2$\\
\ind$\neg P(B\union\set x)$\\
\ind$A:\set{}$\\
\ind$A=\emptyset$\\
\ind$O:\set{0,2}$\\
\ind$B\not\subseteq O$\\
$\mathcal O: \set{\set{0,2},\set1}$\\
$a:2$\\
$A:\set{}$\\
\nnd$A:\set{}$\\
\ind$B:\set{2}$\\
\ind$\mathcal O:\set{\set{0,2},\set1}$\\
\ind$A=\emptyset$\\
\ind$O:\set{0,2}$\\
\ind$B\subseteq O$\\
$\mathcal O: \set{\set{0,2},\set1}$\\

\end{document}