\documentclass{article}

\usepackage{amsmath,amssymb,graphicx,algpseudocode,algorithm,amsthm}
\usepackage[margin=1in]{geometry}
\usepackage{mathrsfs}
\let\mathcrl\mathscr
\usepackage[mathscr]{euscript}
\usepackage{marginnote}
\usepackage{hyperref}
\usepackage{qtree}
\usepackage{graphicx}
\usepackage{tikz}
\geometry{reversemarginpar}


\author{Benji Altman}

\def\latex{\LaTeX\ }

\def\useLim{\limits}
\newcommand{\question}[1]{\marginnote{#1}}
\let\union\cup
\let\inter\cap
\let\emptyset\varnothing
\let\bigunion\bigcup
\let\biginter\bigcap
\let\composed\circ
\let\cross\times
\def\And{\textit{ and }}
\def\Or{\textit{ or }}
\def\sbSeperator{\,\middle|\,}
\def\Return{\State\textbf{return}\par}
\def\ZNonNegative{{\mathbb Z_{\ge 0}}}
\newcommand{\setcomp}[1]{{#1}^{\mathsf{c}}}
\newcommand{\prodfrom}[3]{\prod\useLim_{#1}^{#2}\LB {#3} \RB}
\newcommand{\sumfrom}[3]{\sum\useLim_{#1}^{#2} \LB {#3} \RB}
\newcommand{\unionfrom}[3]{\bigunion\useLim_{#1}^{#2} \LB {#3} \RB}
\newcommand{\interfrom}[3]{\biginter\useLim_{#1}^{#2} \LB {#3} \RB}
\newcommand{\interacross}[2]{\interfrom{#1}{}{#2}}
\newcommand{\unionacross}[2]{\unionfrom{#1}{}{#2}}
\newcommand{\sumacross}[2]{\sumfrom{#1}{}{#2}}
\newcommand{\prodacross}[2]{\prodfrom{#1}{}{#2}}
\newcommand{\Lim}[3]{\lim\useLim_{{#1} \to {#2}}\LB {#3} \RB}
\newcommand{\set}[1]{\left\{ {#1} \right\}}
\newcommand{\setbuilder}[2]{\left\{{#1} \sbSeperator {#2}\right\}}
\newcommand{\derivative}[2]{\frac{d}{d{#2}}\LB {#1} \RB}
\newcommand{\Exists}[2]{\exists_{#1}\LB {#2} \RB}
\newcommand{\All}[2]{\forall_{#1}\LB {#2} \RB}
\newcommand{\abs}[1]{\left|{#1}\right|}
\newcommand{\card}[1]{\left| {#1} \right|}
\newcommand{\range}[1]{\textit{\textbf{Rng}}\left( {#1} \right)}
\newcommand{\domain}[1]{\textit{\textbf{Dom}}\left( {#1} \right)}
\newcommand{\pset}[1]{\mathcal P\left( {#1} \right)}
\newcommand{\pair}[2]{\left( {#1} , {#2} \right)}
\def\closure{\overline}
\newcommand{\limpts}[1]{{#1} '}
\newcommand{\ooint}[2]{\left( {#1} , {#2} \right)}
\newcommand{\ocint}[2]{\left( {#1} , {#2} \right]}
\newcommand{\coint}[2]{\left[ {#1} , {#2} \right)}
\newcommand{\ccint}[2]{\left[ {#1} , {#2} \right]}
\newcommand{\eqclass}[1]{\bar{#1}}
\newcommand{\ceil}[1]{\left\lceil {#1} \right\rceil}
\newcommand{\floor}[1]{\left\lfloor {#1} \right\rfloor}
\newcommand{\inv}[1]{{#1}^{-1}}
\def\true{\text{True}}
\def\false{\text{False}}
\newcommand{\ball}[2]{B_{#1}\left({#2}\right)}
\def\LB{}
\def\RB{}
\newcommand{\cannonicalSet}[1]{\left[ #1 \right]}
\let\lxor\oplus
\newcommand{\norm}[1]{\left|\left|{#1}\right|\right|}

\newtheorem{theorem}{Theorem}[section]
\theoremstyle{definition}
\newtheorem{definition}{Definition}[section]

\title{A Guided Tour to Turing Completeness}


\begin{document}
	\maketitle
	\begin{abstract}
		The common consensus within the mathematical and computational communities is that there are no models for computation that can do something a Turing machine can not. In this paper we look at different models of computation that are equally powerful to a Turing machine and ones that aren't as powerful in an attempt to find some of the fundamental building blocks that make something Turing complete.
	\end{abstract}
	
	\section{Introduction}
	In computer science we have an idea called Turing Completeness. In order to fully understand this we first need to examine what a Turing machine is. Our goal however isn't to just explain Turing Completeness, but rather to understand what the building blocks of a Turing complete system is, and in doing this we will want to explore multiple Turing Complete systems and some systems that are not quite Turing Complete in order explore their similarities and differences.
	
	\subsection{Terminology}
	Before we can get started we need to define the notion of a Language. 
	
	Let $\Sigma$ be a finite set, we may call it our alphabet. We say $x$ is a character or $x$ is a symbol if $x \in \Sigma$.\footnote{symbol and character are used interchangeably in this paper and have the exact same meaning unless otherwise denoted.} A string on $\Sigma$ is defined as a finite sequence of symbols. Finally A language $L$ on $\Sigma$ is defined to be a set of strings.
	
	For notational reasons we write strings somewhat differently then we usually would a finite sequence. For example, let $\Sigma = \set{1,2,3}$, then $1123$ is the same as the finite sequence $(1,1,2,3)$. This can be confusing if for example $1123\in\Sigma$. To avoid this issue we will write $1123 = (1,1,2,3)$ and $\{1123\}$ is the string with just the single element $1123$. In this paper we will also never have a symbol that could lead to any ambiguity. It is also worth noting that the string with no elements exists. We will, unless otherwise stated, denote this with $\lambda$; this will also be called the empty string.
	
	Finally, if $s$ is a string we may refer to it's first character as $s_0$ and it's second as $s_1$ and so on. For the entirety of this paper all counting starts at $0$ as it is considered a natural number.\footnote{\url{https://www.cs.utexas.edu/users/EWD/transcriptions/EWD08xx/EWD831.html}}
	
	\section{Deterministic Finite Automata}
	The first machine we investigate is a Deterministic Finite Automata, or DFA for short. A DFA has a set of states, a set of instructions. One state must be a special start state and at least one state must be an acceptance state. Instructions have three parts: a symbol, a start state, and an end state. When the DFA is given a string it starts at it's start state, $q_0$. If there is an instruction that has $q_0$ as the start state and whatever the first character in the string is, then we go to the end state given in the instruction and start again with the first character of the string removed and starting in our new state. If there is no instruction then the DFA rejects the string. 
	
	For example, if I am given the string $aab$ and my DFA has states $\set{q_0,q_1}$ with $q_0$ being the start state and $q_1$ being the only acceptance state. Additionally it has instructions:
	\begin{center}
		\begin{tabular}{c|c|c|c}
			\#&Start State & Character & End State\\
			\hline
			0&$q_0$ & $a$ & $q_0$ \\
			1&$q_0$ & $b$ & $q_1$ \\
			2&$q_1$ & $b$ & $q_1$ \\
		\end{tabular}
	\end{center}
	Then the DFA works as follows
	\begin{enumerate}
		\item Start at $q_0$ (the start state) with string $aab$. $a$ is the character so we use instruction 0 and go to state $q_0$.
		\item We now are at $q_0$ and we have the string $ab$ and we again go to $q_1$ by instruction 0.
		\item We now are at $q_0$ and we have the string $b$ and we now use instruction 1 to go to $q_1$.
		\item We now have an empty string, we are on state $q_1$ which is an acceptance state so this string is accepted.
	\end{enumerate}
	
	Every DFA defines a language. The language it defines is the set of strings that the DFA accepts. Any language that has a DFA is called a regular language. For example any finite language (a language with only finitely many strings) is regular as we may make a DFA that accepts every individual string in the language. The language $\set{\lambda, a, aa, aaa. \ldots}$ is regular as we will have only a single state $q_0$ that is both an acceptance state and the start state and the only instruction will be $q_0$ goes to $q_0$ when an $a$ is seen.
	
	There is a corresponding notation that actually makes all this quite easy, called regular expressions. There are two kinds of regular expression: there is the formal mathematical regular expression that we will explore momentarily, and the computer programming regular expression that is based of the former, however is far more complicated.
	
	We define a regular expression to be read as follows. You may have a single symbol $a$, this is a valid regular expression and represents the language containing only $a$. If $p$ and $q$ are regular expressions then $p$
	
	\begin{itemize}
	\item If $a$ is a symbol, then $a$ is a valid regular expression representing the language $\set{a}$. That is the language with only $a$ the string $a$.
	\item Recalling that $\lambda$ represents the empty string, $\lambda$ is a valid regular expression representing the language $\set{\lambda}$, that is the language with only the empty string.
	\item Let $p$ and $q$ be valid regular expressions with $p$ representing languages $P$ and $Q$ respectively. We say $(p)+(q)$ is a valid regular expression representing the language $P\union Q$.
	\item Let $p$ and $q$ be valid regular expressions with $p$ representing languages $P$ and $Q$ respectively. We say $(p)(q)$ is a valid regular expression. In order to define what it represents we will adopt the notation that for strings $a$ and $b$, $ab$ is a $a$ concatenated with $b$. Now the language defined by $(p)(q)$ will be $\setbuilder{ab}{a \in P \land b \in Q}$.
	\item Let $p$ be a valid regular expression representing language $P$. We say $(p)^*$ is a valid regular expression representing language $K$, which we define recursively as $K = \set{\lambda} \union \setbuilder{ab}{a\in K\land b\in P}$.
	\end{itemize}
	
	We now make it easier to write by taking out the necessity for all the parentheses by adding an order of operations.
	\begin{enumerate}
		\item The Kleene star: $a^*$
		\item Juxtaposition: $ab$
		\item Addition: $a+b$
	\end{enumerate}
	This means the DFA we gave as an example above, may also be represented as $a^*b^*$, why this is left to the reader as a fun exercise. 
	
	\section{Push Down Automata}
	Now it's worth noting that many every day languages are regular. For example, telephone numbers or email addresses. However the language of all valid regular expressions is not itself regular. The reason for this is the matching of parentheses. Take a simpler example, the language $a^nb^n$. This would be the language $\set{\lambda, ab, aabb, aaabbb, \ldots}$. Now a regular expression has no idea of memory so this language must not be regular. Here we introduce a stack.
	
	A stack is an object that allows for two fundamental operations. Push and Pop. If we have a stack $S$ and we push $a$ onto $S$ then $a$ has been added to the top of $S$. When we Pop from $S$ whatever is on top of $S$ is removed and returned. This means to see what is at the bottom of a stack, or indeed how many items are in a stack, one must throw away all the items. This is a very limited form of memory, however with it we can now tackle problems like $a^nb^n$ or even the language of all valid regular expressions.
	
	We construct a Push Down Automata, or PDA for short, from a DFA. We add to our DFA a stack, and every instruction may be based on what is on top the stack and also may give us stack instructions. The stack instructions must be a finite list of pops or pushes. The stack starts out empty, and it may contains only symbols.
	
	This defines a new class of language, a Context Free Grammar. Now any Regular Language is a Context Free Grammar as the PDA could just ignore it's stack, however let us construct a PDA for $a^nb^n$ to show we can do something new. Let our state set be $\set{q_0, q_1,q_2,q_3}$ with $q_0$ and $q_3$ being acceptance states and $q_0$ the start state. We also allow $\lambda$ to represent what is read when the stack is empty, this will be important.
	\begin{center}
		\begin{tabular}{c|c|c|c|c|c}
			\#&Start State & Character & Top of Stack & End State &Stack Operations\\
			\hline
			0&$q_0$ & $a$ & $\lambda$ & $q_1$ & \\
			1&$q_1$ & $a$ & $\lambda$ & $q_1$ & $\textbf{Push}(a)$ \\  
			2&$q_1$ & $a$ & $a$ & $q_1$ & $\textbf{Push}(a)$ \\
			3&$q_1$ & $b$ & $\lambda$ & $q_3$ & \\
			4&$q_1$ & $b$ & $a$ & $q_2$ & $\textbf{Pop}$ \\
			5&$q_2$ & $b$ & $a$ & $q_2$ & $\textbf{Pop}$ \\
			6&$q_2$&$b$&$\lambda$&$q_3$&
		\end{tabular}
	\end{center}
	In order to test this let try out this machine on the string $aabb$.
	\begin{enumerate}
		\item We start with the string $aabb$ and at state $q_0$, and our stack is empty. As such we use instruction 0, and we simply go to state $q_1$.
		\item We are at state $q_1$ with an empty stack and string $abb$. We then follow instruction 1, staying at state $q_1$ and pushing an $a$ onto the stack.
		\item We now have the string $bb$, while at state $q_1$ and having stack $a$. Due to this we follow instruction 4, go to state $q_2$ and pop off the top of the stack.
		\item We now have the string $b$, with empty stack and at state $q_2$. We must now follow instruction 6 and go to state $q_3$. We have now gone through the string and have ended on an acceptance state.
	\end{enumerate}
	
	Now, Push Down Automata are much more powerful then a regular language. There is a lot that can be done with just a stack. For example, given a specific alphabet $\Sigma$, the language of all regular expressions on that alphabet has a push down automaton:
	\begin{center}
		\begin{tabular}{c|c|c|c|c|c}
			\#&Start State & Character & Top of Stack & End State &Stack Operations\\
			\hline
			0&$q_0$ & \verb|(| & $\lambda$ & $q_0$ & $\textbf{Push}(\verb|(|)$ \\
			
			1&$q_0$ & $\sigma \in \Sigma$ & $\lambda$ & $q_1$&\\ 
			2&$q_0$ & \verb|(| & \verb|(| & $q_0$&$\textbf{Push}(\verb|)|)$\\
			3&$q_0$ & $\sigma \in \Sigma$ & \verb|(| & $q_2$&\\
			4&$q_0$ & \verb|(| & \verb|)| & $q_0$ & $\textbf{Push}(\verb|)|)$ \\
			5&$q_0$ & $\sigma\in\Sigma$ & \verb|)| & $q_0$ & \\
			6&$q_1$ & + & $\lambda$ & $q_0$ & \\
			7&$q_1$ & $^*$ & $\lambda$ & $q_1$&\\
			8&$q_1$ & $\sigma \in \Sigma$ & $\lambda$ & $q_1$&\\
			9&$q_1$ & \verb|(| & $\lambda$ &$q_2$& $\textbf{Push}(\verb|(|)$ \\
			10&$q_2$ & + & \verb|(| & $q_0$& \\
			11&$q_2$ & $^*$ & \verb|(| & $q_2$&\\
			12&$q_2$ & $\sigma \in \Sigma$ & \verb|(| & $q_2$&\\
			13&$q_2$ & \verb|(| & \verb|(| &$q_2$& $\textbf{Push}(\verb|)|)$\\
			14&$q_2$ & \verb|)| & \verb|(| &$q_1$&$\textbf{Pop}$\\
			15&$q_2$ & + & \verb|)| & $q_0$& \\
			16&$q_2$ & $^*$ & \verb|)| & $q_2$&\\
			17&$q_2$ & $\sigma \in \Sigma$ & \verb|)| & $q_2$&\\
			18&$q_2$ & \verb|(| & \verb|)| &$q_2$& $\textbf{Push}(\verb|)|)$\\
			19&$q_2$ & \verb|)| & \verb|)| &$q_2$&$\textbf{Pop}$\\
			20&$q_0$ & $\lambda$ & $\lambda$ & $q_1$ &\\
			21&$q_0$ & $\lambda$ & \verb|(| & $q_2$&\\
			22&$q_0$ & $\lambda$ & \verb|)| & $q_2$&\\
			23&$q_1$&$\lambda$&$\lambda$&$q_1$&\\
			24&$q_2$&$\lambda$&\verb|(|&$q_2$&\\
			25&$q_2$&$\lambda$&\verb|)|&$q_2$&
		\end{tabular}
	\end{center}\footnote{Here where we expect the character lambda (instructions 20-25), we actually mean to use lambda as a symbol as it is needed in regular expressions, however when we say top of stack has a lambda, that is the stack is empty.}
	Where $q_0$ is the start state and $q_1$ is the only acceptance state.
	
	Now one might logically ask, what happens if we add another stack to this machine, or change it from a stack to a queue\footnote{A queue is like a stack, but symbols are added at the bottom instead of at the top.}. In either case we actually end up with something that is equivalent to a Turing machine.
	
	\section{Turing Machine}
	A Turing machine has a `tape' that is infinitely long in both directions. Each spot on the tape can either be empty, or may contain a symbol. We start out with the tape empty, except that the string that we operate on is written onto the tape. For example if we were operating on $abc$ then the tape would be $(\ldots,\lambda,\lambda,a,b,c,\lambda,\lambda,\ldots)$. The Turing machine has a head (much like a typewriter or 3D printer) which can move along the tape. The head starts at the beginning of the input string (in our example it would start at the spot with $a$). Now each machine has a set of states, much like before. The only difference with our states is that we do not have acceptance states, rather this a a single Halt state that ends the machine. To test for acceptance we print a response onto the tape and leave the head at the beginning of the response. Our instructions will be similar to before, with the components being: Start State, Character at Head, End State, Character to Write, Direction to Move. The start and end state are just like before, with the special condition that end state may be the special Halt. The character to read and write may be any character or an empty ($\lambda$). The direction to move may either be left, right, or stay; this will tell the head where to move on the tape. If at any point in the execution of the program, there is no instruction that is valid, halt immediately.
	
	Turing machines aren't simple and even their smaller examples can still be quite large. The special about a Turing machine is that it can validate nearly any language you can think of. The set of languages it can validate are called computable. Additionally the Turing Machine can do much more then that. It can give outputs that aren't just true or false. Simply by writing something the tape it can give any output at all. You can construct arithmetic on anything you have a way to represent within the Tape using a Turing Machine. This is most rigorously put in the Church Turing Theses
	
	
	
	\comment{
	\begin{center}
		\begin{tabular}{c|c|c|c|c|c}
			\#&Start State & Character at Head & End State & Character to Write & Direction to Move\\
			\hline
			0&$q_0$ & $\lambda$ & Halt & $a$ & Stay \\
			1&$q_0$ & $a$ & $q_1$ & $a$ & Right \\
			2&$q_1$ & $a$ & $q_1$ & $a$ & Right \\
			3&$q_1$ & $b$ & $q_2$ & $b$ & Right \\
			4&$q_2$ & $a$ & Halt & $\lambda$ & Stay \\
			5&$q_2$ & $b$ & $q_2$ & $b$ & Right \\
			6&$q_2$ & $c$ & $q_3$ & $c$ & Right \\
			7&$q_3$ & $a$ & Halt & $\lambda$ & Stay \\
			8&$q_3$ & $c$ & $q_3$ & $c$ & Right\\
			9&$q_3$ & $\lambda$ & $q_4$ & $a$ & Left \\
			10&$q_4$ & $a$ & $q_4$ & $a$ & Left \\
			11&$q_4$ & $b$ & $q_4$ & $b$ & Left \\
			12&$q_4$ & $c$ & $q_4$ & $c$ & Left \\
			13&$q_4$ & $\lambda$ & $q_5$ & $a$ & Left\\
			14&$q_5$ & $\lambda$ & $q_6$ & $c$ & Right\\
			15&$q_6$ & $a$ & $q_7$ & $a$ & Right \\
			16&$q_7$ & $a$ & $q_8$ & $\lambda$ & Right\\
			17&$q_8$ & $a$ & $q_9$ & $a$ & Right\\
			18&$q_9$ & $\lambda$ & $q_9$ &$\lambda$&Right\\
			19&$q_9$ & $b$ & $q_{10}$ & $\lambda$ & Right\\
			20&$q_9$ & $a$ & Halt & $\lambda$ & Stay\\
			21&$q_{10}$ & $a$ & Halt & $\lambda$& Stay\\
			22&$q_{10}$ & $b$ & $q_{10}$ & $b$& Right\\
			23&$q_{10}$ & $c$ & $q_{11}$ & $\lambda$& Left\\
			24&$q_{11}$ & $a$ & $q_{12}$ & $a$ & Left\\
			25&$q_{11}$ & $b$ & $q_{13}$ & $b$ & Left\\
			26&$q_{11}$ & $\lambda$ & $q_{11}$ & $\lambda$ & Left \\
			27&$q_{12}$ & $a$ & Halt & $\lambda$ & Stay \\
			28&$q_{12}$ & $\lambda$ & Halt & $\lambda$ & Stay \\
			29&$q_{12}$ & $c$ & $q_{15}$ & $\lambda$ & Right \\
			30&$q_{13}$ & $b$ & $q_{13}$ & $b$ & Left \\
			31&$q_{13}$ & $a$ & $q_{14}$ & $a$ & Left \\
			32&$q_{13}$ & $\lambda$ & $q_{13}$ & $\lambda$ & Left \\
		\end{tabular}
	\end{center}
}
	\begin{quote}
		``A function on the natural numbers is computable by a human being following an algorithm, ignoring resource limitations, if and only if it is computable by a Turing machine.'' \\
		----- Wikipedia
	\end{quote}
	
	This is saying that any algorithm you can do can be done by a Turing Machine. Now whats remarkable about this statement is that despite a lack of proof, it has been accepted nearly universally by academia. So far there has been no reason to doubt this. Here is where the term Turing Complete comes in.
	
	\begin{definition}
		A system is Turing Complete if it can simulate any Turing Machine.
	\end{definition}
	
	For example, the system called Turing Machine is Turing Complete (trivially so), however a specific Turing Machine may not be, as it can only do one task. There are however Universal Turing Machines that take in another Turing Machine (formated as a string in some way), followed by some input, and can simulate the Turing Machine on that input. These Universal Turing Machines are themselves also Turing Complete.
	
	\section{Other Turing Complete Systems}
	Turing Completeness is the highest level of computable power that a system can have, despite this Turing Complete systems pop up in games and other places by accident all the time. For example Magic the Gathering, a popular card game, can implement any Turing Machine with up to a certain number of states on a alphabet with a fairly small number of characters. There is a Universal Turing machine that fits within these requirements and as such Magic the Gathering is Turing Complete. We won't be looking at Magic's Turing Machine as it's convoluted and uninteresting, it simply is an example of how even in unexpected places Turing Completeness pops up. Another good example of this is C++, a programming language. C++ was by all means meant to be Turing Complete, and it is, however within C++ there is a system called Templates. This template system is not so much part of the executing code and is rather just things that are done as the code is turned from C++ into Computer readable binaries. This system is to help with some concepts like abstraction in C++, however it was found that Templates, completely by accident, were also Turing Complete. Rather than looking at these complicated examples, lets look at some simple systems that are Turing Complete.
	
	\subsection{Procedural Languages}
	Procedural (or Imperative) programming Languages are the most standard type of programming language. They tend to achieve Turing Completeness by the use of a few simple components
	
	\begin{itemize}
		\item \textbf{Steps} - A program's most basic feature is stepping through. First line 1 happens, then line 2, and so on. This is very similar how states work in a Turing Machine.
		\item \textbf{Variables} - Variables are a way of holding data that can be changed, this is like the tape in a Turing Machine. Notice it is different then the stack from a Push Down Automata as it can access any information without throwing out data.
		\item \textbf{Loops and Conditionals} - A loop allows one to keep doing the same steps over and over again. This way a program doesn't just run through each step and then end. A loop also must have some way to end on some condition. In it's most basic form these work by having a conditional `jump' statement. It's allows you to go to any line of code on some condition. If the steps in a program are like the Turing Machines' states, then these are like the instructions allowing us to go where we want based on what is in the tape (or variables in this case).
	\end{itemize}
	
	Common languages of this type are: Java, C++, C, JavaScript, Python, R, Assembly Languages, Fortran, Basic.
	
	
	\subsection{Functional Languages}
	Where procedural languages work by stepping through a program, functional languages are completely different. They simply use Recursion to recreate all the things that a Procedural language can do. The basic idea here is that we can get looping by using recursion, and we can also use recursion to create new data, so we don't need loops or variables. Each function call creates new constant values when it is called. We still have conditionals as we will need to have end conditions in our recursion. This is actually how the Templates in C++ get their Turing Completeness.
	
	Common languages of this type are: LISP (and it's descendent's), Haskell, F\#, and Mathematica.
	
	In reality many procedural languages support functional programming and vice versa. For example C++, JavaScript, Python and R can all be functional.
	
	\subsection{Other}
	
	There are other weirder and less common ways to achieve Turing Completeness that are actually used in programming languages. I would rather focus on simple systems that are Turing complete, however not programming languages. The simplest system that one can formalize and find to be Turing Complete are Cellular Automata. John Conway's Game of Life is the most famous and with only a few simple rules on an infinite 2-D board, the Game of Life is Turing Complete. Even more surprising is the Automata Rule 110. This Automata is much like Conway's Game of Life but on a 1-D board, and with even simpler rules.
	
	\section{Conclusion}
	Now we've seen some ways to create Turing Completeness and some methods that don't quite make it, it's hard to say what exactly is needed to get Turing Completeness, but there does seem to be some common themes. The ability to store data and act on any part of that data without having to throw out information seems to be the key. For example in a Push Down Automata with a Queue instead of a Stack, one can simply cycle through the stack as much as they need and find any of the data.\footnote{In the two stack system you can either simulate a queue or go straight to simulating a Turing Machine by treating one stack as the tape to the left of the head and the other stack as what the head is pointing to and everything to the right of that.} It's hard to say if this by itself is enough, and as far as I can tell, there is no formal way to tell if something is Turing Complete other than just implementing a Universal Turing machine in it, but this does seem like a good litmus test. It would be nice to look at more systems that are weaker than a Turing Machine in order to get more contrast.
	
\end{document}